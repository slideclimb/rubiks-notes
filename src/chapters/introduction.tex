We will use the notation used in~\cite[Ch. 14]{Hofstadter}, which is based on the notation in~\cite{FreySingmaster}.
This means that we will use the capital letters \rr U, \rr D, \rr L, \rr R, \rr F, \rr B for the clockwise rotations of the Up, Down, Left, Right, Front, and Back face respectively.
For the counter-clockwise Rotations we will use \rr{Up}, \rr{Dp}, \rr{Lp}, \rr{Rp}, \rr{Fp}, \rr{Bp}.

To refer to a cubie, we will use a tuple or triple of lowercase italic letters, e.g., \cubie{ur} refers to the edge cubie between the Up and the Right face, and \cubie{urf} refers to the corner cubie between the Up, Right, and Front face.
Note that \cubie{ur} and \cubie{ru} refer to the same cubie, but that their orientation is different (they are flipped).
For the corner cubies, we will always put the letters in clockwise order.
That is, we use \cubie{urf} and not \cubie{ufr}.

Instead of refering to a cubie in a certain cubicle, we can also refer to a cubie with a certain \emph{home location}.
For this we will use uppercase italic letters, e.g., $FUR$ for the cubie that belongs in the cubicle between the Front, Up, and Right face.
It is very likely that the $fur$ cubie has a different home location, e.g., $LUF$.