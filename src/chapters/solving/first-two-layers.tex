\section{The First Two Layers}

\subsection{The White Cross}
\RubikFaceUp XWX WWW XWX
\RubikFaceDown XXX XYX XXX
\RubikFaceLeft XBX XBX XXX
\RubikFaceRight XGX XGX XXX
\RubikFaceFront XOX XOX XXX
\RubikFaceBack XRX XRX XXX
\begin{goal}
    Solve the white cross.
    Note that the colours on the side of the cross should match the colours of the centrepieces.
\end{goal}
This can be done using your intuition.

\subsection{White Corners}
\RubikFaceUp WWW WWW WWW
\RubikFaceDown XXX XYX XXX
\RubikFaceLeft BBB XBX XXX
\RubikFaceRight GGG XGX XXX
\RubikFaceFront OOO XOX XXX
\RubikFaceBack RRR XRX XXX
\begin{goal}
    Solve the white face.
\end{goal}

Assume we solve the cubie with home location $FUR$.
If this cubie is not in the bottom layer, and thus in the top layer, we first have to move it to the bottom layer.
Do this by turning the cube so that our cubie is the $fur$ cubie, and doing the sequence \textRubik R, \textRubik{Dp}, \textRubik{Rp}.
Then $fur \to dfr \to dlf$, and we have placed a different cubie in the $FUR$ location.

Now --- our cubie is on the bottom layer --- turn your cube so that the to be solved cubie has as its home location $FUR$.
Turn the bottom layer so that this cubie is in the $FRD$ location (independent of orientation.)
There are three possible orientations for this cubie.

\subsubsection{$drf$ orientation}
Note that the $frd$ orientation is the mirror of this orientation.
\RubikFaceUp XWX WWW XWX
\RubikFaceDown XXX XYX XXX
\RubikFaceLeft XBX XBX XXX
\RubikFaceRight XGX XGX XXX
\RubikFaceFront XOX XOX XXX
\RubikFaceBack XRX XRX XXX
\RubikSliceBottomR XXW GXX
\RubikFaceDown XXO XXX XXX

\textcube{Move $dfr \to dlf$ and $ru \to rf$, to put them next to each other and into place.}
\sequencearrow{Dp, Rp}
\textcube{$dlf \to dfr \to fur$.}
\sequencearrow{D, R}
\notextcube

\subsubsection{$rdf$ orientation}
\RubikFaceUp XWX WWW XWX
\RubikFaceDown XXX XYX XXX
\RubikFaceLeft XBX XBX XXX
\RubikFaceRight XGX XGX XXX
\RubikFaceFront XOX XOX XXX
\RubikFaceBack XRX XRX XXX
\RubikSliceBottomR XXG OXX
\RubikFaceDown XXW XXX XXX

\textcube{Rotate the corner cubie: $rdf \to rbd \to frd$}
\sequencearrow{Rp, Dp}
\textcube{$frd$ cubie is in the wrong orientation, move the edge up without the corner cubie: $frd \to lfd$, $rf \to ru$, $lfd \to frd$.}
\sequencearrow{Dp, R, D}
\textcube{This is the mirror of the $drf$ orientation.}


\subsection{Middle Layer}
\RubikFaceUp WWW WWW WWW
\RubikFaceDown XXX XYX XXX
\RubikFaceLeft BBB BBB XXX
\RubikFaceRight GGG GGG XXX
\RubikFaceFront OOO OOO XXX
\RubikFaceBack RRR RRR XXX
\begin{goal}
    Solve the middle layer.
\end{goal}

First turn the cube upside down, so the white face is on the bottom.
Look for one of the edge cubies of the middle layer that is hanging out on the top layer.
Turn the top layer so that the colour of the edge cubie is matching the colour of a centre cubie on one of the sides.

\RubikFaceUp WWW WWW WWW
\RubikFaceDown XXX XYX XXX
\RubikFaceLeft BBB XBX XXX
\RubikFaceRight GGG XGX XXX
\RubikFaceFront OOO XOX XXX
\RubikFaceBack RRR XRX XXX
\RubikRotation{z2}
\RubikFaceUp XXX XYX XBX
\RubikSliceTopR XOX XXX

\textcube{We want the \cubie{fu} cubie next to the \cubie{frd} cubie. Move \cubie{fu}$\to$\cubie{lu}, then \cubie{frd}$\to$\cubie{urf}.}
\sequencearrow{U, R}
\textcube{Now put them next to each other in the top layer. \cubie{lu}$\to$\cubie{fu} (which will move \cubie{urf}$\to$\cubie{ubr}), move it back: \cubie{ubr}$\to$\cubie{fur}.}
\sequencearrow{Up, Rp}
\textcube{Move this pair to their correct position. First put them next to \cubie{fd}. \cubie{uf}$\to$\cubie{ur}, \cubie{fr}$\to$\cubie{fu}, \cubie{ur}$\to$\cubie{uf}.}
\sequencearrow{Up, Fp, U}
\textcube{And into place. \cubie{fu}$\to$\cubie{fr}.}
\sequencearrow{F}
\notextcube

Putting this all together we have the following sequence:

\RubikFaceUp WWW WWW WWW
\RubikFaceDown XXX XYX XXX
\RubikFaceLeft BBB XBX XXX
\RubikFaceRight GGG XGX XXX
\RubikFaceFront OOO XOX XXX
\RubikFaceBack RRR XRX XXX
\RubikRotation{z2}
\RubikFaceUp XXX XYX XBX
\RubikSliceTopR XOX XXX
\notextcube
\sequencearrow{U, R, Up, Rp, Up, Fp, U, F}
\notextcube

We can use the mirror sequence to move \cubie{fu}$\to$\cubie{fl}.
Do one of these sequences with all the edge cubies until the four of them are in the right place.
Note that we can also use this to get an edge cubie to the top layer so we can put it in its correct location using, again, this sequence.
