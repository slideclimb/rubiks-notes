\documentclass{article}

\usepackage{a4wide}
%\usepackage{rubiksettings}
\usepackage{rubikcube, rubikrotation}
\usepackage{float}
\usepackage{caption}
\usepackage{graphicx}
\usepackage{xparse}

\ExplSyntaxOn
% \CountSubStr{<substring>}{<string>}
\NewDocumentCommand{\CountSubStr}{ m m }{
\seq_set_split:Nnn \l_tmpa_seq { #1 } { #2 }
\int_eval:n {(\seq_count:N \l_tmpa_seq) }
}
\ExplSyntaxOff

% Prints the current state of the cube with given text underneath it.
% The
\newcommand\textcube[1]{
\noindent
\begin{minipage}[t]{4cm}
    \begin{figure}[H]
        \ShowCube{4cm}{1}{\DrawRubikCubeRU}
        \caption*{#1}
    \end{figure}
\end{minipage}
}

\newcommand\miniwidth{}
\newcommand\four{4}
\newcommand{\bloop}{\four cm}
% Print a sequence and an arrow.
\newcommand\sequencearrow[1]{%
    \noindent%
    \renewcommand\miniwidth{\CountSubStr{,}{#1}}%
    \begin{minipage}[t]{\four cm}%
        \quad\ShowSequence{\ }{\Rubik}{#1}{\ $\longrightarrow$ \quad}
        \RubikRotation{#1}
    \end{minipage}
}


\begin{document}



    \RubikCubeSolved
%    \newcommand\test{4cm}
%    \begin{minipage}{\test}
%        \begin{figure}[H]
%            \ShowCube{4cm}{1}{\DrawRubikCubeRU}
%            \caption*{This is the first figure}
%        \end{figure}
%    \end{minipage}
    \textcube{I am some text.
    Saying things so one can remember what to turn next.}
    \sequencearrow{R, U, Rp}
    \textcube{Less text.}
    \sequencearrow{R}
    \textcube{A little text. A bit more text}
    \sequencearrow{U, R, Up, Rp}
    \textcube{Tadaaaaaa!}
    \textcube{Tadaaaaaa!}

\end{document}